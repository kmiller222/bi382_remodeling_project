\PassOptionsToPackage{unicode=true}{hyperref} % options for packages loaded elsewhere
\PassOptionsToPackage{hyphens}{url}
%
\documentclass[]{article}
\usepackage{lmodern}
\usepackage{amssymb,amsmath}
\usepackage{ifxetex,ifluatex}
\usepackage{fixltx2e} % provides \textsubscript
\ifnum 0\ifxetex 1\fi\ifluatex 1\fi=0 % if pdftex
  \usepackage[T1]{fontenc}
  \usepackage[utf8]{inputenc}
  \usepackage{textcomp} % provides euro and other symbols
\else % if luatex or xelatex
  \usepackage{unicode-math}
  \defaultfontfeatures{Ligatures=TeX,Scale=MatchLowercase}
\fi
% use upquote if available, for straight quotes in verbatim environments
\IfFileExists{upquote.sty}{\usepackage{upquote}}{}
% use microtype if available
\IfFileExists{microtype.sty}{%
\usepackage[]{microtype}
\UseMicrotypeSet[protrusion]{basicmath} % disable protrusion for tt fonts
}{}
\IfFileExists{parskip.sty}{%
\usepackage{parskip}
}{% else
\setlength{\parindent}{0pt}
\setlength{\parskip}{6pt plus 2pt minus 1pt}
}
\usepackage{hyperref}
\hypersetup{
            pdftitle={Remodeling Project Paper},
            pdfauthor={Kirsten Miller},
            pdfborder={0 0 0},
            breaklinks=true}
\urlstyle{same}  % don't use monospace font for urls
\usepackage[margin=1in]{geometry}
\usepackage{graphicx,grffile}
\makeatletter
\def\maxwidth{\ifdim\Gin@nat@width>\linewidth\linewidth\else\Gin@nat@width\fi}
\def\maxheight{\ifdim\Gin@nat@height>\textheight\textheight\else\Gin@nat@height\fi}
\makeatother
% Scale images if necessary, so that they will not overflow the page
% margins by default, and it is still possible to overwrite the defaults
% using explicit options in \includegraphics[width, height, ...]{}
\setkeys{Gin}{width=\maxwidth,height=\maxheight,keepaspectratio}
\setlength{\emergencystretch}{3em}  % prevent overfull lines
\providecommand{\tightlist}{%
  \setlength{\itemsep}{0pt}\setlength{\parskip}{0pt}}
\setcounter{secnumdepth}{0}
% Redefines (sub)paragraphs to behave more like sections
\ifx\paragraph\undefined\else
\let\oldparagraph\paragraph
\renewcommand{\paragraph}[1]{\oldparagraph{#1}\mbox{}}
\fi
\ifx\subparagraph\undefined\else
\let\oldsubparagraph\subparagraph
\renewcommand{\subparagraph}[1]{\oldsubparagraph{#1}\mbox{}}
\fi

% set default figure placement to htbp
\makeatletter
\def\fps@figure{htbp}
\makeatother


\title{Remodeling Project Paper}
\author{Kirsten Miller}
\date{5/15/2020}

\begin{document}
\maketitle

{
\setcounter{tocdepth}{2}
\tableofcontents
}
\hypertarget{summary-of-the-paper}{%
\subsubsection{Summary of the paper}\label{summary-of-the-paper}}

For my remodeling project, I recreated the work of a 2020 paper titled
``Impact of influenza vaccine-modified infectivity on attack rate, case
fatality ratio and mortality'' by Nah et al.~This paper explores the
potential impacts of vaccination on a hypothetical influenza outbreak.
It is well understood that vaccination typically decreases
susceptibility and reduces the infectious period (leading to a faster
recovery rate). However, the authors also take into consideration recent
work that has observed that vaccination may increase viral shedding and
therefore transmission in vaccinated yet infected individuals. They
model how vaccine-modified effects (susceptibility, infectivity, and
recovery) can change the result of an outbreak. The authors used an SIRV
model, an extension of an SIR model, in order to determine the impact of
vaccination on the outcomes of a hypothetical influenza outbreak.

The study examined the possible outcomes of the outbreak based on
various vaccination characteristics. This involved varying both the
vaccine coverage and vaccine-modified effects and then calculating
measures indicative of the severity of the outbreak (including the
attack rate, final size, case fatality ratio, and death rate). This
study was therefore able to predict the outcomes for a wide variety of
scenarios with different combinations of levels of vaccine coverage and
vaccine-modified effects. These scenarios were visualized through a
variety of figures displaying the outcomes for different combinations of
varying parameter values.

The main finding of the study was that vaccination with a strong vaccine
had a positive effect in reducing the severity of the epidemic, even
with increased viral shedding in vaccinated individuals. Specifically,
based on the results of their model, vaccination usually reduced the
attack rate and final size for the epidemic, so vaccination still seemed
to be found to be productive despite vaccine-modified effects that could
increase viral shedding. Death rate and case fatality ratio also
generally decreased as vaccination coverage increased, except when
vaccine-modified susceptibility was low and vaccination coverage was
high, this could lead to lower deaths but a higher case fatality ratio.

\hypertarget{description-of-the-model}{%
\subsubsection{Description of the
model}\label{description-of-the-model}}

\textbf{Model:} This paper used an SIRV model to model the hypothetical
influenza outbreaks. This is an extension of an SIR model, which is a
compartmental model that uses ordinary differential equations to show
the flow of individuals between different groups (from those who are
susceptible to the disease, to those who are infected, then to those who
recover from a disease). The SIRV model adds classes of susceptible
vaccinated (V) and infected vaccinated (Iv), in addition to the usual
susceptible (S), infected (I), and recovered (R) population groups of an
SIR model. Therefore, it incorporates vaccine-modified effects by
including vaccine-modified infectivity (mi), vaccine-modified
susceptibility (ms), vaccine-modified recovery (mr), and
vaccine-modified mortality (md). The SIRV model uses differential
equations in order to relate continuous rates of change of these
population classes to the biological processes of disease transmission,
as shown in the model equations below:

\textbf{Conditions:} The authors of the paper assume the basic
reproduction number, R0, to be set at a value of 1.5 for all of their
model runs (except for the Figure 1). The initial population size, N0
which is used to set the inital age classes in order to solve the model
equations, is set at 1000 for all of the model runs.

\textbf{Parameters:} Throughout their analysis, the authors of the paper
generally maintain the same values for the base parameters of their
model which includes the recovery rate (\(\gamma\)), disease-induced
mortality (\(\delta\)), and baseline transmission rate (\(\beta\)). They
then vary the effect of the vaccine on the population by changing the
values for vaccination coverage (\(\rho\)), vaccine-modified
susceptibility (ms), vaccine-modified infectivity (mi), and
vaccine-modified recovery (mr).

\textbf{Values of parameters:} The authors of the paper determine
parameter values from the literature. For most of their model runs, they
use a recovery rate (\(\gamma\)) = 1x10-3 and disease-induced mortality
(\(\delta\)) = 1x10-3 or 0. Baseline transmission rate (\(\beta\)) is
estimated using the values for R0, \(\delta\), \(\gamma\), and N0,
calculated using the equation \(\beta= R_0(\delta\gamma/N_0)\). They
vary the vaccine-modified parameters over a range of values, with
vaccine coverage (\(\rho\)) = {[}0, 1{]}, vaccine-modified
susceptibility (ms) = {[}0, 0.6{]}, vaccine-modified infectivity (mi) =
{[}0, 6{]}, and vaccine-modified recovery (mr) is less than or equal to
1.

\textbf{Outputs:} By varying combinations of these parameters, the study
observes how vaccination affects the outcomes of the outbreak. Using the
SIRV model, the focus of the paper was to understand the outcomes and
characterize hypothetical outbreaks with different vaccination
conditions, which is accomplished by calculating measures from the model
output that represent the outcome of the epidemic (final size, attack
rate, case fatality ratio and death rate). The final size is the total
number of infections over the epidemic, which is also recorded for
vaccinated (Fv) and non-vaccinated (Fn) individuals. The attack rate is
the proportion of the population that experiences infection over the
period of the epidemic, which is the sum of the attack rate for
vaccinated individuals (Av) and the attack rate for non-vaccinated
individuals (An). A successful vaccination reduces the overall attack
rate. The case fatality ratio (CFR) is defined as the ratio of deaths
occurring from an influenza infection to the total number of cases. The
final measure is the death rate per 100,000 population.

\hypertarget{methods}{%
\subsubsection{Methods}\label{methods}}

\textbf{Reproducing the model:} I was able to reproduce the paper's SIRV
model by starting with the SIR model we learned in class and building
off of it to add the additional classes of susceptible but vaccinated
individuals (V) and infected but vaccinated individuals (Iv). I was also
then able to easily include the vaccine modified effect parameters (mi,
ms, mr, md), as determined by the equations given in the SIRV model. I
set the initial sizes for each of the population groups (S, V, I, Iv)
using equations given in the paper, and assumed the initial size of the
recovered population to be zero. I set up the model by creating a
function in R for the SIRV equations and ran the function using the
ode() function (using the deSolve package) to solve the ordinary
differential equations and calculate the relevant output values.

\textbf{Reproducing the figures:} In order to reproduce most of the
figures, I looped over the model in order to vary multiple parameters
simultaneously. Most of the figures required a range of vaccination
coverage (\(\rho\)) values from 0 to 1 plotted on the x-axis, for
multiple different values, or a range of values, of one of the other
vaccine-modified effects (mi, ms, mr). Based on these parameter inputs,
the figures required the calculation of one of the output measures (eg.
final size, attack rate, CFR, death rate). I saved the output values for
each iteration of the loop in a vector to use for the plots.

\textbf{Running the model:} I ran the model with the initial number of
infections set at 20 individuals (this was not specified by the paper
but was the number I found to best reproduce the figures). This
information was fed into the model in order to calculate the initial
population sizes for the infected non-vaccinated (I) and infected
vaccinated (Iv) classes, based on the value of vaccination coverage for
the population (\(\rho\)). I ran the model for a time period of at least
100 days, finding that the epidemic seemed to have run its course after
this amount of time. One hundred days was sufficient to successfully
recreate Figures 2, 3, 5, 10, and 11 (although running the model for 200
days also worked for these figures), but 200 days were required to
successfully reproduce Figure 7. For the figures involving the
vaccine-modified viral shedding, mi/mr ratio, I typically set mr at 1
and varied mi in order to vary the mi/mr ratio. Overall, I was
successful in recreating the figures accurately, although there were
some discrepancies between my recreated figures and the original paper.

\hypertarget{figure-by-figure-breakdown}{%
\subsubsection{Figure-by-figure
breakdown}\label{figure-by-figure-breakdown}}

\hypertarget{figure-1}{%
\paragraph{Figure 1}\label{figure-1}}

Figure 1: Relationship between vaccine-modified effects and the epidemic
threshold

I found the value of the epidemic threshold based off of varying values
of ρ and R0, and changing ms and mi/mr. The epidemic threshold is
calculated by \((1 + (1 - R_0) / (ρR_0)\)

Methods: For Figure 1, I actually did not use the full model, but
instead used the following equation:
\(R_v = (1-ρ)R_0 + ((mi*ms)/mr)ρR_0\)

I then rearranged this equation for R0 and plugged in varying values for
ρ, ms, and mi/mr. My recreation of Figure 1 does noticeably differ from
the study. However, this should not have an impact on the remainder of
my results because the calculations involved were isolated from the
model, which I used for the remainder of the figures.

Results: Maintaining Rv \textless{} 1 requires a higher vaccine-modified
susceptibility or infectivity (ms or mi) or a lower vaccine-modified
recovery (mr). Therefore, a balance between ms, mi, and mr is needed.

\hypertarget{figures-2-3}{%
\paragraph{Figures 2 \& 3}\label{figures-2-3}}

Figures 2\&3: Impact of vaccination on attack rate

Methods: I looped through the model for 3 different values of mi/mr
(Figure 2) or ms (Figure 3) as well as varying ρ from 0 to 1, and
calculated the attack rate for each iteration of the loop. I set mr to
1, and varying mi in order to vary the overall mi/mr ratio.

Results: When vaccine-modified viral shedding (mi/mr) \textgreater{} 1,
the attack rate for the overall population decreases as vaccination
coverage increases, but vaccine-modified susceptibility (ms) must also
be small.

\hypertarget{figure-4}{%
\paragraph{Figure 4}\label{figure-4}}

Figure 4: Sign of dA/d\(\rho\), the numerical value above which the
attack rate increases while vaccination coverage (\(\rho\)) increases,
and below which the attack rate decreases as vaccination coverage
increases

Methods: I calculated the value for dA/d\(\rho\) while looping through
the model for varying values of \(\rho\) from 0 to 1 and varying values
of ms from 0 to 0.6 using the following equation: \$dA/d \rho = \$

Results: Figure 4 shows that for a lower ms value, the attack rate will
decrease as vaccination coverage increases, but for a higher ms value,
the attack rate will increase as vaccination coverage increases.

\hypertarget{figure-5}{%
\paragraph{Figure 5}\label{figure-5}}

Figure 5: Attack rates (5a) and final sizes (5b) for vaccinated and
non-vaccinated populations

Methods: I looped through the model while varying ρ from 0 to 1, and
calculated A, Av, \& An (Figure 5a) or F, Fv, \& Fn (Figure 5b) for each
iteration of the loop.

Results: The vaccine-modified susceptibility (ms) \textless{} 1, so
attack rate of vaccinated individuals is less than non-vaccinated
individuals (Av \textless{} An). The attack rate in the total population
(A) decreases with increasing vaccination coverage, but in this case Av
and An increase slightly. While Av \textless{} An (when ms
\textless{}1), the final size of the vaccinated individuals may be
larger than the final size for the non-vaccinated individuals (Fv
\textgreater{} Fn) depending on the value of vaccination coverage.

\hypertarget{figure-7}{%
\paragraph{Figure 7}\label{figure-7}}

Figure 7: Case fatality ratio and death rate

Methods: This was the most difficult figure to recreate (coding-wise)
because I had to vary two variables along a range of values. I looped
through the model for varying values of both ρ and ms (Figure 7a) or mi
(Figure 7b). I calculated CFR or death rate for each iteration of the
loop.

Results: Lower vaccine-modified susceptibility (ms) may lead to lower
deaths but a higher CFR (7a), depending on vaccination coverage.
Vaccination reduces CFR and deaths when vaccine-modified infectivity
(mi) is low (seen in Figure 7b).

\hypertarget{figure-10}{%
\paragraph{Figure 10}\label{figure-10}}

Figure 10: Weak (Type 1) vs.~strong (Type 2) vaccines

Methods: I looped through the model while varying ρ from 0 to 1 for each
type of vaccine. Type 1 (weak) vaccines have lower mi/mr and higher ms
than Type 2. I calculated A, Av, \& An (Fig 5a) or F, Fv, \& Fn (Fig 5b)
for each iteration of the loop.

Results: The average attack rate (A) and final size (F) decrease as
vaccination coverage increases for both Type 1 and 2 vaccines. Av and An
increase for Type 2 vaccines (seen in Figure 10bi), and final size of
the vaccinated population (Fv) increases for both types (seen in Figures
10aii and 10bii).

\hypertarget{figure-11}{%
\paragraph{Figure 11}\label{figure-11}}

Figure 11: Relative risk of infection and severe infection outcome

Methods: I looped through the model while varying ρ from 0 to 1 for each
type of vaccine. I calculated the relative risk of infection (Figure
11a) and or relative risk of severe infection outcome (Figure 11b) for
each iteration of the loop.

Results: The relative risk of severe outcome is lower than relative risk
of infection for both types of vaccines. Both relative risk of infection
and relative risk of severe infection outcome are lower for Type 2
(strong) vaccines.

\hypertarget{discussion}{%
\subsubsection{Discussion}\label{discussion}}

\hypertarget{levins-characterization-of-models-generality-precision-and-realism}{%
\paragraph{1. Levin's characterization of models (generality, precision
and
realism)}\label{levins-characterization-of-models-generality-precision-and-realism}}

(This semester you read and the class discussed Levins' strategy of
model building in population biology, where he characterized models as
generally maximizing two and sacrificing one of the following:
generality, precision, and realism. Please describe how your study fits
into this framework, and justify.)

This study fits into Levins' framework by maximizing generality and
precision while sacrificing realism. The model maintained generality by
showing the results of an outbreak for many different combinations of
varied parameter values. It did this by presenting a multitude of
scenarios in which multiple parameters were varied over a range of
values. This was accomplished by running the model many times with
different conditions and therefore determining the range of possible
outcomes. The model also had a high level of precision because specific
results could be determined for a particular set of parameter values.
Unique values could be chosen for each of the parameters individually,
and exact values describing the epidemic (eg. final size, attack rate)
could be obtained from the model specific to the input. Each point on a
figure in the paper represented a precise scenario and outcome. The
study therefore found very specific results for each aspect of the
hypothetical influenza outbreak that it modeled.

I thought that the area in which the model was most lacking was realism.
While the model did have very precise conditions and demonstrated
generality by showing the results for many precise scenarios, I felt
that there was a disconnect between these scenarios and real life. The
paper did not discuss any potential adjustments that would have to be
made for different real scenarios or different locations, nor did it
consider outside factors that could occur in real outbreaks (such as
location, traveling/rate of contacts, or population density). These
factors could have been acknowledged and accounted for to some extent
through varying the R0 value, but the authors kept R0 set at 1.5 for all
of their simulations except for Figure 1. They did acknowledge that R0
is known to range from 1.47 to 2.27 for influenza, but they did not
adjust this in any of their other simulations after Figure 1. In real
life scenarios, the reproduction number varies by outbreak and also
varies of the course of an outbreak. The R0 value also impacts the
baseline transmission rate (\(\beta\)), so this value was also assumed
to be constant. The study also did not include information about how the
outbreak started, and it was unclear how many infections they assumed at
the start of the outbreak. This would also be important information to
know in applying this model to a real world outbreak.

I think that because their model is very precise but also can be
generalized to many scenarios, it will be possible to apply this work to
real situations, but I still think that they could have addressed this
more directly in the paper. While the study's results were both precise
and able to be generalized, the results felt disconnected from reality.
At the least, I think the study could have done a better job explaining
how their results could apply to real outbreaks of influenza.

\hypertarget{reproducibility}{%
\paragraph{2. Reproducibility}\label{reproducibility}}

(The idea of reproducibility is a cornerstone of scientific
methodologies. Please describe two ways in which your study facilitated
reproducibility and three two that your study could have improved, and
justify.)

This study was very reproducible in the fact that it very clearly
defined the model equations and parameters. The study also did a good
job defining the parameters, including the values and equations used to
calculate values for the parameters. I was able to very closely recreate
the figures because nearly all of the parameters were explicitly defined
for each figure. The study was also reproducible because it used an SIRV
model, an extension of the well-known SIR model, which made it very easy
to understand and recreate for someone who has prior experience with SIR
models for disease.

What was more difficult to reproduce about the study was the exact
conditions under which the authors ran their model. The study did not
discuss how many model runs they performed, how long they assumed the
outbreak to last, or the initial number of infections for the outbreak.
The lack of information in this area not only makes it more difficult
for someone to precisely replicate their model, but also leaves a gap of
information that is important to understanding the significance of their
model to future situations. Understanding how an epidemic started and
how long it lasted is key in interpreting and applying their findings.
To improve reproducibility, the study could have been more specific by
defining the number of model runs, initial number of infections, and
duration of the outbreak that they used for their model (and whether
this varied at all between figures).

\hypertarget{follow-up-study}{%
\paragraph{3. Follow-up study}\label{follow-up-study}}

(In your opinion, what would be the most interesting follow-up study to
yours? Why?)

I think that an interesting and important follow-up study would involve
varying the R0 value along with vaccine-modified effects. It is clear
that the R0 value for a disease has a major impact on the outcomes of an
outbreak and its epidemic potential. It is also known that the R0 value
varies for influenza (as noted by this paper) both by outbreak and over
time during an outbreak. Since this paper almost explicitly modeled with
an R0 value of 1.5, I think it would be important to build upon the work
of this study by varying R0 while investigating the various scenarios.
The study should focus on whether different R0 values could change the
impact of vaccine-modified effects on an outbreak.

In addition, I think that more studies are needed on vaccine-modified
effects. I did not know anything about vaccine-modified effects before
reading this paper. This study cites one prior paper which observed a
potential increase in viral shedding in vaccinated infected individuals.
Therefore, the hypothetical scenarios examined by this study do not seem
to have much of a scientifically investigated basis. More research on
vaccine-modified effects would be important because it would help
determine which scenarios explored by this study are most likely to
actually transpire. Since this study did a good job exploring a range of
possibilities, it is likely that they covered some realistic scenarios,
but it would be important to better understand vaccine-modified effects
to know how to apply the work of this study.

\hypertarget{my-understanding-of-modeling}{%
\paragraph{4. My understanding of
``modeling''}\label{my-understanding-of-modeling}}

(I hope that throughout the course you learned a lot about ecological
``modeling.'' ``Modeling'' is a broad term that can mean many different
things. Please reflect on how your understanding of ``modeling'' has
changed from the beginning of the semester.)

I wish that I had written down what I thought ``modeling'' meant at the
beginning of the semester, but I would guess that my understanding has
changed significantly. Before taking this class, I had a much more vague
idea of what modeling is and how it can be used. Through BI382, I have
been exposed to a wide range of ecological models, from the base models
for density-independent and density-independent growth, to age and stage
structure, to spatial patterns and metapopulation modeling, and finally
and most relevantly to disease modeling.

I now can build off of the basic definition of a model as ``an
abstraction of reality'' to state and understand the importance of
models in ecology and beyond. Instead of considering models to be
abstract and complex mathematical predictions, I now think of models as
a way to describe relationships between entities we observe (and
although complex math is often useful to describe these relationships, a
model involves more than math itself). While I now understand that the
basis of a model is to describe relationships, perhaps the most
important application of models is to determine how these relationships
might change with varying conditions (such as time). And since we often
don't know the exact values that should be used to describe things as
conditions change, models allow us to explore and describe a range of
possibilities as conditions change. Over the course of the semester I
have become more aware of the power and importance of models to be
informative on a variety of scales and subjects. While the focus of this
semester was on modeling ecological subjects, studying a range of
topics, types, complexities of models within the field of ecology has
opened my mind to the flexibility of models and allowed me to realize
how models can be used to describe many types of relationships in many
fields of study.

Although it is technically outside of the main curriculum of BI382, it
would be shortsighted not to acknowledge how my understanding of
modeling has been changed by living through the current COVID-19
pandemic. Taking this class during a time period in which models are
being used to predict our future and influence (or sometimes
unfortunately not influence) imminent important decisions from
policymakers has been incredibly thought-provoking and enlightening as
to the role models play in everyday life. This has demonstrated to me
that although even the best models by the most skilled experts are not
going to be perfectly accurate, modeling still provides important, and
sometimes unexpected, insights that cannot be gleaned from simple
observation. The current use of models for COVID-19 has also highlighted
the extreme fundamental importance of using models for disease ecology
and our current future in light of the pandemic.

In sum, my understanding of models has evolved substantially over the
course of the semester as a result of spending a lot of time learning
about, understanding, and creating models. My perception of models has
been both clarified and broadened as I have gained a better
understanding of what models are and the many ways in which they can be
used. I have learned that while models can range from simple to complex,
general to precise, and realistic to abstract, they are extremely
powerful in their ability to describe and predict current and future
relationships.

\hypertarget{variables-and-parameters}{%
\subsubsection{Variables and
Parameters}\label{variables-and-parameters}}

\hypertarget{state-variables}{%
\subparagraph{State Variables}\label{state-variables}}

\begin{itemize}
\tightlist
\item
  Susceptible (non-vaccinated) population (S)
\item
  Susceptible vaccinated population (V)
\item
  Infected (non-vaccinated) population (I)
\item
  Infected vaccinated population (Iv)
\item
  Recovered population (R)
\item
  Initial population size (N0) = 1000
\end{itemize}

\hypertarget{fixed-parameters}{%
\subparagraph{Fixed parameters}\label{fixed-parameters}}

(Some of these parameters could technically change over time but were
held constant in the model)

\begin{itemize}
\tightlist
\item
  Recovery rate for non-vaccinated individuals (\(\gamma\)) = 0.33
\item
  Disease-induced mortality (\(\delta\)) = 1x10-3 or 0
\item
  Baseline transmission rate (\(\beta\)) \(= R_0[(\delta\gamma/N_0)]\)
\item
  Initial recovered population = 0
\item
  S0, V0, I0, Iv0: calculated based on the number of initial infections
  (sometimes varied because the paper did not specify the number of
  initial infections)
\end{itemize}

\hypertarget{varied-parameters}{%
\subparagraph{Varied parameters:}\label{varied-parameters}}

\begin{itemize}
\tightlist
\item
  Vaccine coverage (\(\rho\)) = {[}0, 1{]}
\item
  Vaccine-modified susceptibility (ms) = {[}0, 0.6{]}
\item
  Vaccine-modified infectivity (mi) = {[}0, 6{]}
\item
  Vaccine-modified recovery (mr) (less than or equal to 1)
\item
  Vaccine-modified mortality (md) = {[}0, 1{]}
\item
  Relative risk of severe infection outcome (Msio) = {[}0, 1{]}
\item
  Relative risk of severe infection outcome in vaccinated individuals to
  non-vaccinated individuals (RRsio) = Calculated
\item
  Duration of epidemic = {[}100, 200{]} (varied because the paper did
  not specify)
\end{itemize}

\hypertarget{paper-citation}{%
\paragraph{Paper Citation}\label{paper-citation}}

Nah, K., Alavinejad, M., Rahman, A., Heffernan, J. M., \& Wu, J. (2020).
Impact of influenza vaccine-modified infectivity on attack rate, case
fatality ratio and mortality. Journal of Theoretical Biology, 492,
110190.

\end{document}
